\section{Autoren}

Der Artikel wurde in Zusammenarbeit von fünf Yahoo! Researchern verfasst. Sie haben ihre Arbeit auf dem ersten ACM Symposium für Cloud Computing\footnote{url{http://research.microsoft.com/en-us/um/redmond/events/socc2010/}} eingereicht. Im Folgenden werden die fünf Autoren kurz vorgestellt.


\subparagraph{Brian Frank Cooper}\footnote{\url{http://www.brianfrankcooper.net/}}
war zum Zeitpunkt der Veröffentlichung von \cite{ycsb10} „Principal Research Scientist“ bei Yahoo!. Sein wissenschaftlicher Werdegang begann an der Universität Stanford und setzte sich an der Universität Georgia fort, wo er Assistenzprofessor war. Mittlerweile ist er Entwickler bei Google. Brian Cooper hat bei vielen Konferenzen um das Thema Cloud mitgewirkt und diverse wissenschaftliche Arbeiten veröffentlicht.

\subparagraph{Adam Silberstein}\footnote{\url{https://www.linkedin.com/in/adamsilberstein}}
ist ebenfalls ein ehemaliger Yahoo! Researcher. Seinem LinkedIn Profil zufolge ist er ein praktisch orientierter Wissenschaftler, der Hands-on Bibliotheken und Frameworks entwirft. Unterarderem war er auch maßgeblich an der Entstehung des YCSB Benchmarking Tools beteiligt. Mittlerweise arbeitet er bei Trifacta.

\subparagraph{Erwin Tam}
hat ebenfalls den Arbeitgeber gewechselt und ist nun Entwickler bei Google. Zuvor war er Researcher bei Yahoo und war an der Veröffentlichung einiger weniger Paper beteiligt. 

\subparagraph{Raghu Ramakrishnan}\footnote{\url{https://www.linkedin.com/in/raghu-ramakrishnan-0059733}}
war für 4 Jahre ein führender Wissenschaftler bei Yahoo! und arbeitet mittlerweile bei Microsoft als „CTO for Data“. Vor seiner wirtschaftlichen Karriere war er 22 Jahre Professor and der Universität Wisconsin. Er hat an diversen wissenschaftliche Publikationen mitgewirkt und wurde laut Google Scholar 32633 mal zitiert\footnote{\url{https://scholar.google.com/citations?user=udZSrkYAAAAJ}}.

\subparagraph{Russell Sears}\footnote{\url{http://rsea.rs/}}
ist ein aktiver Entwickler im Cloud und BigData Umfeld. Er studierte an der UC Berkeley und war Researcher bei Yahoo! und Microsoft. Momentan arbeitet er als Wissenschaftler in einem hochspezialisierten Unternehmen, das performante Speichermedien für BigData-Zwecke herstellt. Laut Semantic Scholar wurde er ca. 3258 mal zitiert\footnote{\url{https://www.semanticscholar.org/author/Russell-Sears/3086388}}.


