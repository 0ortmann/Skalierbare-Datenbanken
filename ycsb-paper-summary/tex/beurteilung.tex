\section{Beurteilung}

Der Artikel ist meiner Meinung nach sehr nachvollziehbar motiviert. Es wird zunächst eine einfache Definition von Cloud Serving Systemen gegeben und die klare Anforderung dargelegt, diese Systeme vergleichbar zu machen. Anschließend wird knapp dargelegt, welche groben Aspekte ein solches System klassifizieren (könnten). Ferner wird der Leser motiviert, eigene Vergleichskriterien solcher Systeme zu definieren; das YCSB Framework ermöglicht dies praktischer Weise.

Mit dieser Motivation wird YCSB vorgestellt, als \textit{die} Lösung der Wahl um solche Vergleiche zu ermöglichen. Die dargestellten Interna des Frameworks versprechen dabei einfache Nutzbarkeit und Anpassungsmöglichkeiten.

Die Autoren arbeiten sehr zielführend; in einem optimistisch anmutenden Stil beschreiben sie Schwächen, die durch den Einsatz von YCSB in einigen der getesteten Storage Lösungen gefunden werden konnten. Dabei wird deutlich, dass YCSB vielmehr das Auffinden solcher Probleme ermöglicht und nicht deren Lösung verspricht.

Insgesamt wird dem Leser der Eindruck vermittelt, mit YCSB ein Produkt an der Hand zu haben, das als neues universelles Benchmarking Tool in der Datanbanken Bewegung genutzt werden kann.

Es bleibt eine kleine Kritik offen: Die Autoren haben spezialiesierte Hardware verwendet, um die Tests durchzuführen. Dies ist im Grunde genommen auch ausreichend; es wäre meiner Meinung nach sinnvoller gewesen, hier Standard Hardware zu verwenden, um praxisnahere Ergenisse zu präsentieren.

Allgemein ist das Paper sehr aussagekräftig und verdeutlicht den Bedarf und die Vorteile eines uniformen Benchmarking Tools für Cloud Serving Systeme.