\section{Autoren}

Der Artikel wurde in Zusammenarbeit von insgesamt neun Wissenschaftlern der Universität von Kalifornien, Berkeley erarbeitet. Auf der neunten \textit{USENIX Conference on Networked Systems Design and Implementation} wurde die Arbeit veröffentlicht. Im Folgenden werden alle Autoren kurz vorgestellt und ihr fachlicher Bezug zum Thema des Artikels knapp umrissen.

\subparagraph{Matei Zaharia} \footnote{https://people.csail.mit.edu/matei/} ist Assistenzprofessor bei CSAIL\footnote{MIT: Computer Science and Artificial Intelligence Laboratory} und Co-Founder der Firma Databricks. Während seiner Doktorarbeit an der UC Berkeley war er unteranderem maßgeblich an der Entwicklung von \textit{Spark} beteiligt. Er wurde laut Google Scholar insgesamt ca. 19880 mal zitiert.

\subparagraph{Mosharaf Chowdhury} \footnote{http://www.mosharaf.com/} ist Assistenzprofessor bei der Universität Michigan, Department Computer Science and Engineering.\footnote{http://cse.umich.edu/}. Auch er schrieb seine Doktorarbeit an der UC Berkeley. Momentan forscht er im Bereich Big Data und Cloud Computing. Laut Google Scholar wurde Chowdhury ca. 5720 mal zitiert.


\subparagraph{Tathagata Das}  ist zur Zeit bei Databricks angestellt, wo er am Spark Projekt und \textit{Spark Streaming} arbeitet. Er hat seinen Master Abschluss an der UC Berkeley gemacht und war an der Entstehung von Spark beteiligt\footnote{https://www.linkedin.com/in/tathadas}. Insgesamt wurde er 2336 mal zitiert (Google Scholar).


\subparagraph{Ankur Dave} \footnote{http://ankurdave.com/} befindet sich momentan im dritten Jahr seines Doktoriats an der UC Berkeley. Er war drei Jahre lang wissenschaftlicher Mitarbeiter dort und während dieser Zeit arbeitete er mit an dem Spark Projekt. Mittlerweile hat er bei fast allen namhaften Bigdata Firmen Praktika absolviert. Laut Google Scholar wurde er 1361 mal zitiert. 


\subparagraph{Justin Ma} \footnote{https://amplab.cs.berkeley.edu/author/jma/} ist seit 2012 Angestellter bei Google. Davor war er graduierter wissenschaftlicher Mitarbeiter an der UC Berkeley. Mittlerweile liegt sein Schwerpunkt mehr auf Netzwerksicherheit als auf Bigdata.\footnote{https://www.linkedin.com/in/justin-ma-73897b2} Laut SemanticScholar wurde er ca. 3470 mal zitiert.\footnote{https://www.semanticscholar.org/author/Justin-Ma/2932392}


\subparagraph{Murphy McCauley} war Master Student an der UC Berkley, wo sein Fokus auf Software Networking lag.\footnote{https://www.usenix.org/system/files/login/articles/zaharia.pdf}


\subparagraph{Michael J. Franklin} \footnote{https://people.eecs.berkeley.edu/~franklin/} ist Professor for Computer Science und Bigdata an der UC Berkley. Seine drei Forschungsschwerpunkte sind Datenbanken, Betriebssysteme und Netzwerke. Er wurde laut Google Scholar ca. 4701 mal zitiert.

\subparagraph{Scott Shenker} \footnote{https://www.eecs.berkeley.edu/Faculty/Homepages/shenker.html} ist Professor an der UC Berkeley und eine Koryphäe in der Bigdata-Szene. Er hält viele Ehrenämter an der UC Berkeley. Laut Google Scholar wurde er ca. 102547 mal zitiert und ist einer der fünf meist zitiertesten Wissenschaftler weltweit. Er war unter anderem Doktorvater von Matei Zaharia und vielen anderen.


\subparagraph{Ion Stoica} \footnote{http://people.eecs.berkeley.edu/~istoica/} ist Professor für Computer Science an der UC Berkeley. Er war maßgeblich beteiligt an der Entstehung von Spark und war Doktorvater für Matei Zaharia und andere der Autoren. Momentan arbeitet auch er bei Databricks. Laut ResearchGate\footnote{https://www.researchgate.net/profile/Ion\_Stoica/citations} wurde er ca. 30231 mal zitiert.



