% !TeX spellcheck=de_DE
% !TeX program=xelatex
\documentclass[12pt]{article}
\PassOptionsToPackage{hyphens}{url}
\usepackage{newclude}
\usepackage{hyperref}
\usepackage[babelshorthands]{polyglossia}
\usepackage{fontspec}
\usepackage{array}
\usepackage[fleqn]{amsmath}
\usepackage{amsfonts, amssymb}
\usepackage[a4paper, hmargin=2.5cm, vmargin=2.0cm]{geometry}
\usepackage{enumerate}
\usepackage{xspace}
\usepackage{graphicx}
\usepackage{xcolor}
\usepackage{enumerate}
\usepackage{fontawesome}
\usepackage{subfigure}
\usepackage{wrapfig}
\usepackage{todonotes}
\usepackage{titlesec}
\usepackage{scrpage2}

\setmainlanguage{german}

\makeatletter
\def\email#1{{\tt#1}}
\def\subtitle#1{\gdef\@subtitle{#1}}
\def\institute#1{\gdef\@institute{#1}}
\def\authors#1#2{\author{#2}}
\makeatother
\title{Review -- „Resilient Distributed Datasets: A Fault-Tolerant Abstraction for In-Memory Cluster Computing“ }
\subtitle{Seminararbeit zu der Veranstaltung \\Skalierbare Datenbanken}
\authors{Felix~Ortmann}{Felix~Ortmann\\ \email{\{0ortmann\}@informatik.uni-hamburg.de}}
\institute{Universität Hamburg\\Fachbereich Informatik\\Vogt-Kölln-Straße 30\\22527 Hamburg}



\newcolumntype{x}[1]{>{\centering\arraybackslash}m{#1}}

\def\calibri#1{\begingroup\fontspec{Calibri}\selectfont#1\endgroup}

\newcommand{\subfigureautorefname}{Abb.}
\renewcommand{\figureautorefname}{Abbildung}
\renewcommand{\sectionautorefname}{Abschnitt}
\renewcommand{\subsectionautorefname}{Unterabschnitt}

\newtheorem{satz}{Satz}[section]
\newtheorem{defi}{Definition}[section]
\newtheorem{korol}{Korollar}[section]

\linespread{1.1}

\pagestyle{scrheadings}
\clearscrheadings
\ofoot{\pagemark}

\titleformat{\section}[hang]{\fontsize{16pt}{16pt}\selectfont\bf}{\thesection\quad}{0pt}{}{}
\titlespacing*{\section}{0pt}{24pt}{6pt}
\titleformat{\subsection}[hang]{\fontsize{14pt}{14pt}\selectfont\bf}{\thesubsection\quad}{0pt}{}{}
\titlespacing*{\subsection}{0pt}{18pt}{6pt}
\titleformat{\subsubsection}[hang]{\fontsize{12pt}{12pt}\selectfont\bf}{\thesubsubsection\quad}{0pt}{}{}
\titlespacing*{\subsubsection}{0pt}{12pt}{6pt}

\pagenumbering{arabic}

\begin{document}
\urlstyle{same}
	
\makeatletter
\def\@maketitle{%
	\newpage
	\null
	\vskip 2em%
	\begin{center}%
		\let \footnote \thanks
		{\LARGE\bfseries \@title \par}%
		\vskip .5em%
		{\Large \@subtitle \par}%
		\vskip 1.5em%
		{\large
			\lineskip .5em%
			\begin{tabular}[t]{c}%
				\@author
			\end{tabular}\par}%
		\vskip 1em%
		{\large \@date}%
	\end{center}%
	\par
	\vskip 1.5em}
\makeatother

\maketitle
\begin{abstract}
	% Motivation bzw. Zusammenfassung
	Dieses Review von „HAT, Not CAP: Towards Highly Available Transactions“ \cite{hnc13} stellt zunächst die Autoren des Artikels kurz vor und setzt sie in Relation zu dem Thema des Artikels. Es folgt eine knappe inhaltliche Zusammenfassung des Artikels sowie eine kritische Beurteilung.
	
\end{abstract}

\include*{tex/autoren}
\include*{tex/inhalt}
\include*{tex/beurteilung}

\nocite{*}
\bibliographystyle{abbrv}
\bibliography{bib}
\end{document}
