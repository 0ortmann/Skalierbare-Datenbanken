\section{Autoren}

Der Artikel wurde in Zusammenarbeit von fünf Wissenschaftlern der Univeritäten Berkeley, Sydney und dem KTH/Royal Institute of Technology verfasst. Die Arbeit wurde auf der 14. USENIX Konferenz 2013 eingereicht, veröffentlicht in den Tagungsberichten / „in proceedings“ der HotOS'13.

\subparagraph{Peter Bailis}\footnote{\url{http://www.bailis.org/}}
ist assistenz Professor für Computer Science an der UC Berkeley. Dort hat er auch seinen Doktorgrad erworben, neben einigen Auszeichnungen für besondere Arbeiten als Researcher. Peter Bailis Forschungsschwerpunkte liegen derzeit auf „large scale data management, distributed protocol design, and architectures for high-volume complex decision support“. Laut Google Scholar wurde er ca. 663 mal zitiert.

\subparagraph{Alan Fekete}\footnote{\url{http://sydney.edu.au/engineering/it/~afek5227/}}
ist derzeit Professor für \textit{Enterprise Software Systems} an der Universität Sydney. Promoviert hat er an der Harvard Universität in Mathematik. Bereits seit 1988 ist er an der Universität Sydney als Forscher und Dozent tätig. Laut Google Scholar wurde er ca. 3825 mal zitiert.

\subparagraph{Ali Ghodsi}\footnote{\url{http://people.eecs.berkeley.edu/~alig/}}
ist derzeit außerordentlicher Professor an der UC Berkeley. Neben seinen akademischen Tätigkeiten ist er Mitgründer und CEO von Databricks. Er promovierte am Royal Institute of Technology. Ghodsi hat an einigen Werken rund um das Thema verteilte Datenbanken mitgewirkt; ebenso an Apache Spark und Apache Mesos.  
Laut Google Scholar wurde er ca. 4801 mal zitiert.

\subparagraph{Joseph M. Hellerstein}\footnote{\url{http://db.cs.berkeley.edu/jmh/}}
ist ebenfalls Professor an der Universität Berkeley und arbeitet als Chief Strategy Officer bei Trifacta. Seine Arbeit wurde mehrfach geehrt, unteranderem von der ACM und dem MIT. Laut Google Scholar wurde er ca. 33054 mal zitiert.

\subparagraph{Ion Stoica} \footnote{http://people.eecs.berkeley.edu/~istoica/} ist Professor für Computer Science an der UC Berkeley. Er leistete zahlreiche bedeutende wissenschaftliche Beiträge und war Doktorvater für einige bekannte Wissenschaftler aus dem Gebiet der verteilten Systeme und Datenbanken, so wie etwa Matei Zaharia. Laut ResearchGate\footnote{\url{https://www.researchgate.net/profile/Ion_Stoica/citations}} wurde er ca. 41053 mal zitiert.