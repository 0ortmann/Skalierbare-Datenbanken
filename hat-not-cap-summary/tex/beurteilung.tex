\section{Beurteilung}

Dem kritischen Leser wird in der recht lang gehaltenen Motivation sehr feinfülig vermittelt, dass das CAP Theorem hochverfügbare Systeme und ACID Eigenschaften nicht in Widerspruch zueinander stellt. Dieser Umstand wird meiner Meinung nach sehr gut dargelegt. Ohne diesen Grad an Ausführlichkeit in der Motivation würde der Artikel nicht viel hermachen.

Die Grundlagen von CAP, den ACID Eigenschaften und Verfügbarkeit im allgemeinen werden gut verständlich dargestellt. Allerdings scheinen mir die Konzepte, die die Autoren in Section 3 beschreiben wenig neu zu sein. Dirty Reads etwa durch Caching zu verhindern und damit dann hohe Verfügbarkeit zu gewährleisten ist in meinen Augen keine bahnbrechende Erkenntnis. 

Der gesamte Artikel rankt sich um die konkreten Fallbeispiele „read committed“ und „ANSI SQL repeatable read“. Meines Empfindens präsentieren die Autoren keine neuartigen oder bahnbrechenden Ideen, so wie der Leser es vermittelt bekommt; viel mehr beschreiben sie ein paar Gedanken zu dem Thema. Es scheint mir ein wenig übertrieben, dass die Autoren diese Überlegungen mit „highly available Transactions“ betiteln.

Alles in allem ein mitreißend geschriebener Artikel der zwar zum Nachdenken einlädt aber wenig neues präsentiert.